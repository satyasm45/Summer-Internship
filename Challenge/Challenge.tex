\documentclass[journal,12pt,twocolumn]{IEEEtran}

\usepackage{setspace}
\usepackage{gensymb}

\singlespacing


\usepackage[cmex10]{amsmath}

\usepackage{amsthm}

\usepackage{mathrsfs}
\usepackage{txfonts}
\usepackage{stfloats}
\usepackage{bm}
\usepackage{cite}
\usepackage{cases}
\usepackage{subfig}

\usepackage{longtable}
\usepackage{multirow}

\usepackage{enumitem}
\usepackage{mathtools}
\usepackage{steinmetz}
\usepackage{tikz}
\usepackage{circuitikz}
\usepackage{verbatim}
\usepackage{tfrupee}
\usepackage[breaklinks=true]{hyperref}
\usepackage{graphicx}
\usepackage{tkz-euclide}
\usepackage{float}

\usetikzlibrary{calc,math}
\usepackage{listings}
    \usepackage{color}                                            %%
    \usepackage{array}                                            %%
    \usepackage{longtable}                                        %%
    \usepackage{calc}                                             %%
    \usepackage{multirow}                                         %%
    \usepackage{hhline}                                           %%
    \usepackage{ifthen}                                           %%
    \usepackage{lscape}     
\usepackage{multicol}
\usepackage{chngcntr}

\DeclareMathOperator*{\Res}{Res}

\renewcommand\thesection{\arabic{section}}
\renewcommand\thesubsection{\thesection.\arabic{subsection}}
\renewcommand\thesubsubsection{\thesubsection.\arabic{subsubsection}}

\renewcommand\thesectiondis{\arabic{section}}
\renewcommand\thesubsectiondis{\thesectiondis.\arabic{subsection}}
\renewcommand\thesubsubsectiondis{\thesubsectiondis.\arabic{subsubsection}}


\hyphenation{op-tical net-works semi-conduc-tor}
\def\inputGnumericTable{}                                 %%

\lstset{
%language=C,
frame=single, 
breaklines=true,
columns=fullflexible
}
\begin{document}


\newtheorem{theorem}{Theorem}[section]
\newtheorem{problem}{Problem}
\newtheorem{proposition}{Proposition}[section]
\newtheorem{lemma}{Lemma}[section]
\newtheorem{corollary}[theorem]{Corollary}
\newtheorem{example}{Example}[section]
\newtheorem{definition}[problem]{Definition}

\newcommand{\BEQA}{\begin{eqnarray}}
\newcommand{\EEQA}{\end{eqnarray}}
\newcommand{\define}{\stackrel{\triangle}{=}}
\bibliographystyle{IEEEtran}
\providecommand{\mbf}{\mathbf}
\providecommand{\pr}[1]{\ensuremath{\Pr\left(#1\right)}}
\providecommand{\qfunc}[1]{\ensuremath{Q\left(#1\right)}}
\providecommand{\sbrak}[1]{\ensuremath{{}\left[#1\right]}}
\providecommand{\lsbrak}[1]{\ensuremath{{}\left[#1\right.}}
\providecommand{\rsbrak}[1]{\ensuremath{{}\left.#1\right]}}
\providecommand{\brak}[1]{\ensuremath{\left(#1\right)}}
\providecommand{\lbrak}[1]{\ensuremath{\left(#1\right.}}
\providecommand{\rbrak}[1]{\ensuremath{\left.#1\right)}}
\providecommand{\cbrak}[1]{\ensuremath{\left\{#1\right\}}}
\providecommand{\lcbrak}[1]{\ensuremath{\left\{#1\right.}}
\providecommand{\rcbrak}[1]{\ensuremath{\left.#1\right\}}}
\theoremstyle{remark}
\newtheorem{rem}{Remark}
\newcommand{\sgn}{\mathop{\mathrm{sgn}}}
\providecommand{\abs}[1]{\vert#1\vert}
\providecommand{\res}[1]{\Res\displaylimits_{#1}} 
\providecommand{\norm}[1]{\lVert#1\rVert}
%\providecommand{\norm}[1]{\lVert#1\rVert}
\providecommand{\mtx}[1]{\mathbf{#1}}
\providecommand{\mean}[1]{E[ #1 ]}
\providecommand{\fourier}{\overset{\mathcal{F}}{ \rightleftharpoons}}
%\providecommand{\hilbert}{\overset{\mathcal{H}}{ \rightleftharpoons}}
\providecommand{\system}{\overset{\mathcal{H}}{ \longleftrightarrow}}
	%\newcommand{\solution}[2]{\textbf{Solution:}{#1}}
\newcommand{\solution}{\noindent \textbf{Solution: }}
\newcommand{\cosec}{\,\text{cosec}\,}
\providecommand{\dec}[2]{\ensuremath{\overset{#1}{\underset{#2}{\gtrless}}}}
\newcommand{\myvec}[1]{\ensuremath{\begin{pmatrix}#1\end{pmatrix}}}
\newcommand{\mydet}[1]{\ensuremath{\begin{vmatrix}#1\end{vmatrix}}}
\numberwithin{equation}{subsection}
\makeatletter
\@addtoreset{figure}{problem}
\makeatother
\let\StandardTheFigure\thefigure
\let\vec\mathbf
\renewcommand{\thefigure}{\theproblem}
\def\putbox#1#2#3{\makebox[0in][l]{\makebox[#1][l]{}\raisebox{\baselineskip}[0in][0in]{\raisebox{#2}[0in][0in]{#3}}}}
     \def\rightbox#1{\makebox[0in][r]{#1}}
     \def\centbox#1{\makebox[0in]{#1}}
     \def\topbox#1{\raisebox{-\baselineskip}[0in][0in]{#1}}
     \def\midbox#1{\raisebox{-0.5\baselineskip}[0in][0in]{#1}}
\vspace{3cm}
\title{Challenge Problem 5}
\author{Satya Sangram Mishra}
\maketitle
\newpage
\bigskip
\renewcommand{\thefigure}{\theenumi}
\renewcommand{\thetable}{\theenumi}
Download all python codes from 
\begin{lstlisting}
https://github.com/satyasm45/Summer-Internship/tree/main/Challenge5/Codes
\end{lstlisting}
%
and latex-tikz codes from 
%
\begin{lstlisting}
https://github.com/satyasm45/Summer-Internship/tree/main/Challenge5
\end{lstlisting}
%
\section{Challenge Question 5}
Express the axis of a parabola in terms of $\vec{V}$,$\vec{u}$,f in general.
%
\section{Explanation}
We have seen in earlier assignment that the equation for parabola having focus $\vec{F}$ and directrix $\vec{n}^T\vec{x}=c$ assuming $\lambda=\norm{\vec{n}}^2$ will be: 
\begin{align}
\vec{x}^T(\lambda\vec{I}-\vec{n}\vec{n}^T)\vec{x}+2(c\vec{n}-\lambda\vec{F})^T\vec{x}+\lambda\norm{\vec{F}}^2-c^2&=0\label{eq:8}
\end{align}

Comparing with standard equation 
\begin{align}
\vec{V}=\lambda\vec{I}-\vec{n}\vec{n}^T\\
\vec{u}=c\vec{n}-\lambda\vec{F}\\
f=\lambda\norm{\vec{F}}^2-c^2
\end{align}
Now we have
\begin{align}
    \lambda\vec{I}-\vec{n}\vec{n}^T=\vec{V}\\
    \vec{n}^T\norm{\vec{n}}^2\vec{I}-\vec{n}^T\vec{n}\vec{n}^T=\vec{n}^T\vec{V}\\
    \norm{\vec{n}}^2(\vec{n}^T-\vec{n}^T)=\vec{n}^T\vec{V}\\
    \vec{n}^T\vec{V}=0
\end{align}
Let:
\begin{align}
    \vec{V}=\myvec{\vec{v_1}&\vec{v_2}}\\
    \implies\vec{n}^T\vec{v_1}=0,\vec{n}^T\vec{v_2}=0
\end{align}
It is also known that atleast one of $\vec{v_1},\vec{v_2}\neq\myvec{0\\0}$. WLOG let $\vec{v_1}\neq\myvec{0\\0}$
The axis of parabola is perpendicular to directrix.
$\therefore$ normal of axis is given as $\vec{v_1}$.
The axis passes through focus,so equation of axis is:
\begin{align}
    \vec{v_1}^T\vec{x}=\vec{v_1}^T\vec{F}\label{eq:1}
\end{align}
We have:
\begin{align}
  c\vec{n}-\lambda\vec{F}=\vec{u}\\
  c\vec{n}^T-\lambda\vec{F}^T=\vec{u}^T\\
  c\vec{n}^T\vec{V}-\lambda\vec{F}^T\vec{V}=\vec{u}^T\vec{V}\\
  0-\lambda\vec{F}^T\vec{V}=\vec{u}^T\vec{V}\\
  -\lambda\myvec{\vec{F}^T\vec{v_1}&\vec{F}^T\vec{v_2}}=  \myvec{\vec{u}^T\vec{v_1}&\vec{u}^T\vec{v_2}}\\
  \vec{v_1}^T\vec{F}=-\frac{\vec{u}^T\vec{v_1}}{\lambda}\label{eq:2}
\end{align}
Also
\begin{align}
    \lambda\vec{I}-\vec{n}\vec{n}^T=\vec{V}\\
    \lambda\vec{V}-\vec{n}\vec{n}^T\vec{V}=\vec{V}^2\\
    \lambda\vec{V}=\vec{V}^2\label{eq:3}
\end{align}
From \ref{eq:1},\ref{eq:2} and \ref{eq:3} the equation of axis is given by:
\begin{align}
    \vec{v_1}^T\vec{x}=-\frac{\vec{u}^T\vec{v_1}}{\lambda}\\
    \text{where ,}\lambda\vec{V}=\vec{V}^2
\end{align}
It is interesting to note that equation is independent of 'f'.
\end{document}
